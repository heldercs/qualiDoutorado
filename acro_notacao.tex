%============================Acronimos e Notação =================================================

\chapter*{Lista de Acrônimos e Notação}

\begin{tabular}{ll}
LMI  & Linear Matrix Inequality (desigualdade matricial linear)\\
LFT  & Linear Fractional Transformation (transformação linear fracionária)\\
LPV  & Linear Parameter-Varying (linear com parâmetros variantes)\\
IQC  & Integral Quadratic Constraint (restrição de integral quadrática)\\
\end{tabular}

\vspace*{1cm}

\begin{tabular}{ll}
$\star$ & indica bloco simétrico nas LMIs\\
$L > 0$ & indica que a matriz $L$ é simétrica definida positiva\\
$L \geq 0$ & indica que a matriz $L$ é simétrica semi-definida positiva\\
$A$ & notação para matrizes (letras maiúsculas do alfabeto latino)\\
$A'$ & ($'$), pós-posto a um vetor ou matriz, indica a operação de transposição\\
$\reais$ & conjunto dos números reais\\
$\mathbb{Z}$ & conjunto dos números inteiros\\
$\mathbb{Z}_+$ & conjunto dos números inteiros não negativos\\
$\mathbb{N}$ & conjunto dos números naturais (incluindo o zero)\\
$\I$ & matriz identidade de dimensão apropriada\\
$\Z$ & matriz de zeros de dimensão apropriada\\
$g!$ & símbolo (!), denota fatorial, isto é, $g!=g (g-1) \cdots (2) (1)$ para $g \in \mathbb{N}$\\
$N$ & especialmente utilizada para denotar o número de vértices de um politopo\\
$n$ & especialmente utilizada para representar a ordem uma matriz quadrada\\
$\simplex$ & simplex unitário de $N$ variáveis\\
$\alpha$ & especialmente utilizada para representar as incertezas de um sistema
\end{tabular}

%==============================================================================================
