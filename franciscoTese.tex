\documentclass[12pt,oneside,final]{book}

\usepackage[portuges,brazil]{babel}
\usepackage[utf8]{inputenc}


\usepackage{indentfirst}
\usepackage{ae}
\usepackage{harvard}
\usepackage{amssymb,fancyhdr,fancybox,epsfig,psfrag,amsmath,tabularx}
\usepackage[paperwidth=8.5in,paperheight=11in,hmargin={25mm,20mm},vmargin={20mm,20mm}]{geometry} %tamanho letter
\usepackage{multirow}
\usepackage[table]{xcolor}

\usepackage[color]{showkeys}
\definecolor{refkey}{rgb}{0.39,0.58,1}
\definecolor{labeled}{rgb}{1,0,0}
\usepackage[Lenny]{fncychap}
\setlength{\headheight}{15pt}

%=========================================== Headers =========================================
\renewcommand{\chaptermark}[1]{\markboth{\chaptername\ \thechapter. \ #1}{ }}
\renewcommand{\sectionmark}[1]{\markright{\thesection. \ #1}}
\fancyhead{}
\fancyfoot{}
\fancyhead[LO]{\nouppercase{\leftmark}}
\fancyhead[RO]{\thepage}

% Espacos H2, Hoo, L2, Loo
\newcommand\Hi{{\mathcal{H}}_{\infty}}
\newcommand\Hd{{\mathcal{H}}_{2}}
\newcommand\Li{{\mathcal{L}}_{\infty}}
\newcommand\Ld{{\mathcal{L}}_{2}}


\def\reais{{\rm I\kern-.17em R}} % R de reais
\def\Dest{{\rm I\kern-.17em D}} % R de reais
\newcommand{\Dgest}{\ensuremath{\mathcal{D}}}
\newcommand{\naturais}{\mathbb{N}}
\newcommand{\inteiros}{\mathbb{Z}_{+}}
\newcommand{\matlab}{{\sc Matlab}}

%super parenteses
\makeatletter
\def\Biggg#1{{\hbox{$\left#1\vbox to29\p@{}\right.\n@space$}}}
\newdimen\bracketwidth
\settowidth{\bracketwidth}{\Biggg(} \makeatother


%\newcommand{\carinha}{\raisebox{-.2ex}{\epsfxsize 3mm \epsffile{cara.eps}}}

% Bolds
\newcommand{\I}{{\bf I}}
\newcommand{\Z}{{\bf 0}}
\newcommand{\Bfres}{\mathcal{B}}

% funcoes
\newcommand{\Tr}{\mbox{Tr}}

% Referencia com parenteses
\newcommand{\bref}[1]{\mbox{(\ref{#1})}}

% Ambiente de equacoes
\newcommand\be{\begin{equation}}
\newcommand\ee{\end{equation}}
\newcommand\vi{\vspace{\baselineskip}}

% newtheorems e similares

\newtheorem{teorema}{\noindent{\bf Teorema} }[chapter]
\newtheorem{lema}{\noindent{\bf Lema} }[chapter]
\newtheorem{corolario}{\noindent{\bf Corolário} }[chapter]
\newenvironment{prova}{\noindent{\bf Prova:}}{\null\hfill $\rule{1.5mm}{1.5mm}$}
\newtheorem{definicao}{\noindent{\bf Definição} }[chapter]

\newcommand{\simplex}{\Delta_N}



\newcommand{\Aal}{A(\alpha)}
\newcommand{\Bal}{B(\alpha)}
\newcommand{\Cal}{C(\alpha)}
\newcommand{\Dal}{D(\alpha)}


\newcommand{\Pal}{P(\alpha)}
\newcommand{\Gal}{G(\alpha)}
\newcommand{\Hal}{H(\alpha)}
\newcommand{\Qal}{Q(\alpha)}
\newcommand{\Xal}{X(\alpha)}
\newcommand{\Xual}{X_1(\alpha)}
\newcommand{\Xdal}{X_2(\alpha)}

\newcommand{\Sfr}{\mathcal{S}}

\newcommand{\Xfal}{\mathcal{X}(\alpha)}
\newcommand{\Bfal}{\mathcal{B}(\alpha)}
\newcommand{\Qfal}{\mathcal{Q}(\alpha)}
\newcommand{\Sfal}{\mathcal{S}(\alpha)}

\newcommand{\Kfr}{\mathcal{K}}
\newcommand{\Ifr}{\mathcal{I}}
\newcommand{\Afr}{\mathcal{A}}
\newcommand{\Pfr}{\mathcal{P}}
\newcommand{\Cfr}{\mathcal{C}}
\newcommand{\Gfr}{\mathcal{G}}

%makes the system use abbreviated citation
\citationmode{abbr}


\begin{document}

\pagenumbering{roman}
\pagestyle{plain}

%================================================================================================
%====================================== FOLHA DE ROSTO ==========================================
%================================================================================================

\begin{center}
\large Universidade Estadual de Campinas\\
Faculdade de Engenharia Elétrica e de Computação
\end{center}

\vspace*{1.5cm}
\begin{center}
\large Francisco Helder Candido dos Santos Filho
\end{center}


\vspace*{2.3cm}

\begin{center}
{\sc  Uma Proposta de Modelo para Comunicação Machine-To-Machine em Redes IoT Heterogêneas}
\end{center}

\vspace*{3.0cm}

\begin{flushright}
\begin{minipage}{9.0cm}
Qualificação apresentada à Faculdade de Engenharia Elétrica e de Computação como  parte dos requisitos exigidos para a obtenção do título de Doutor em Engenharia Elétrica. Área de concentração: Telecomunicações. 

\vspace*{0.5cm}
Orientador: Paulo Cardieri

\end{minipage}
\end{flushright}

\null \vfill

\vspace*{0.5cm}

\begin{center}
Campinas\\2018
\end{center}

\baselineskip 1.1 \baselineskip

%================================= Resumo e Abstract ========================================
\chapter*{Resumo}


\begin{quotation}
\noindent A principal contribuição desta tese é a proposta de uma
metodologia para solução de desigualdades matriciais lineares dependentes de parâmetros que
freqüentemente aparecem em problemas de análise e controle robusto
de sistema lineares com incertezas na forma politópica. O método consiste na parametrização
das soluções em termos de polinômios homogêneos com coeficientes matriciais de grau
arbitrário. Para a construção dessas soluções, um procedimento baseado em resoluções de
problemas de otimização na forma de um número finito de desigualdades matriciais lineares é
proposto, resultando em seqüências de
relaxações que convergem para uma solução polinomial homogênea sempre que uma solução
existe. Problemas de análise
robusta e custo garantido são analisados em detalhes
tanto para sistemas a tempo contínuo quanto para sistemas
discretos no tempo. Vários exemplos numéricos são apresentados
ilustrando a eficiência dos métodos propostos em termos da
acurácia dos resultados e do esforço computacional quando comparados
com outros métodos da literatura.

\vspace*{0.5cm}

\noindent Palavras-chave: Sistemas lineares incertos. Domínio politópico. Estabilidade robusta. Normas
H-2 e H-infinito. Funções de Lyapunov polinomiais homogêneas. Desigualdades matriciais lineares.
Lema de Finsler. Teorema de Pólya. Relaxações convergentes.

\end{quotation}


\chapter*{Abstract}


\begin{quotation}


\noindent This thesis proposes, as main contribution, a new methodology to solve
parameter-dependent linear matrix inequalities which frequently appear in robust
analysis and control problems of linear system with
polytopic uncertainties. The proposed method relies on the parametrization of
the solutions in terms of homogeneous polynomials of arbitrary degree with matrix valued coefficients.
For constructing such solutions, a procedure based on
optimization problems formulated in terms of a finite number of linear
matrix inequalities is proposed, yielding sequences
of relaxations which converge to a homogeneous polynomial solution whenever
a solution exists. Problems of robust analysis and
guaranteed costs are analyzed in details for continuous and
discrete-time uncertain systems. Several numerical examples are
presented illustrating the efficiency of the proposed methods in
terms of accuracy and computational burden when compared to other
methods from the literature.

\vspace*{0.5cm}

\noindent Key-words: Uncertain linear systems. Polytopic domains. Robust stability. H-2 and
H-infinity norms. Homogeneous polynomial Lyapunov functions. Linear matrix inequalities.
Finsler's Lemma. Pólya's Theorem. Convergent relaxations.

\end{quotation}

\null




%=============================== lista de tabelas e figuras ==========================
\listoffigures

\listoftables

%============================ acrônimos, símbolos e notações ========================
%============================Acronimos e Notação =================================================

\chapter*{Lista de Acrônimos e Notação}

\begin{tabular}{ll}
LMI  & Linear Matrix Inequality (desigualdade matricial linear)\\
LFT  & Linear Fractional Transformation (transformação linear fracionária)\\
LPV  & Linear Parameter-Varying (linear com parâmetros variantes)\\
IQC  & Integral Quadratic Constraint (restrição de integral quadrática)\\
\end{tabular}

\vspace*{1cm}

\begin{tabular}{ll}
$\star$ & indica bloco simétrico nas LMIs\\
$L > 0$ & indica que a matriz $L$ é simétrica definida positiva\\
$L \geq 0$ & indica que a matriz $L$ é simétrica semi-definida positiva\\
$A$ & notação para matrizes (letras maiúsculas do alfabeto latino)\\
$A'$ & ($'$), pós-posto a um vetor ou matriz, indica a operação de transposição\\
$\reais$ & conjunto dos números reais\\
$\mathbb{Z}$ & conjunto dos números inteiros\\
$\mathbb{Z}_+$ & conjunto dos números inteiros não negativos\\
$\mathbb{N}$ & conjunto dos números naturais (incluindo o zero)\\
$\I$ & matriz identidade de dimensão apropriada\\
$\Z$ & matriz de zeros de dimensão apropriada\\
$g!$ & símbolo (!), denota fatorial, isto é, $g!=g (g-1) \cdots (2) (1)$ para $g \in \mathbb{N}$\\
$N$ & especialmente utilizada para denotar o número de vértices de um politopo\\
$n$ & especialmente utilizada para representar a ordem uma matriz quadrada\\
$\simplex$ & simplex unitário de $N$ variáveis\\
$\alpha$ & especialmente utilizada para representar as incertezas de um sistema
\end{tabular}

%==============================================================================================


%=============================== sumário =============================================
\tableofcontents


\pagestyle{fancy}

\chapter{Objetivos e Justificativa do Projeto de Pesquisa}



\chapter{Revis\~{a}o Bibliogr\'{a}fica}

Exemplo de citação: \cite{Lya92}

\chapter{Metodologia Utilizada}

Exemplo de citação: \cite{holler14}
\chapter{Plano de Trabalho e Cronograma}

\section*{Plano de Trabalho}

\section*{Cronograma}


%Tabela
\begin{table}[h]
  \centering
	\caption{table}{Cronograma de Atividades.}
	\begin{tabular}{|l|r||l|l||l|l|l|l||l|l|l|l||l|l|l|l|}
		\hline
		\multirow{2}{*}{\textbf{Atividades}} &\footnotesize\textit{Ano}\normalsize & \multicolumn{2}{c||}{2016} & \multicolumn{4}{c||}{2017} & \multicolumn{4}{c||}{2018} & \multicolumn{4}{c|}{2019} \\
		\cline{2-16}
		&\footnotesize \textit{Trimestre}\footnotesize & \scriptsize 3 &\scriptsize 4 &\scriptsize 1 &\scriptsize 2 &\scriptsize 3 &\scriptsize 4 &\scriptsize 1 &\scriptsize 2 &\scriptsize 3 &\scriptsize 4 &\scriptsize 1 &\scriptsize 2 &\scriptsize 3 &\scriptsize 4 \\
		\hline \hline 
	  \multicolumn{2}{|l||}{\footnotesize Créditos em disciplinas} & \cellcolor{gray} & \cellcolor{gray} &  &  & \cellcolor{gray} & \cellcolor{gray} &  &  &  &  &  &  &  & \\
		\hline
		\multicolumn{2}{|l||}{\footnotesize Levantamento bibliográfico} & \cellcolor{gray} & \cellcolor{gray} & \cellcolor{gray} & \cellcolor{gray}  & \cellcolor{gray} & \cellcolor{gray} &  &  &  &  &  &  &  &  \\
		\hline
	\end{tabular}
	\label{tab:cronograma}
\end{table}


\chapter{Resultados e Conclus\~{o}es Parciais}


\subsection*{Resultados Parciais}


\subsection*{conclus\c{c}\~{o}es Parciais}







%=============================== Bibliografia ===============================================
\addcontentsline{toc}{chapter}{Bibliografia}
\renewcommand{\bibname}{Bibliografia}
\markboth{Bibliografia}{Bibliografia}

\bibliographystyle{dcu}
\bibliography{meubib}

\end{document}
