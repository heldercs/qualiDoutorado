\documentclass[12pt,oneside,final]{book}

\usepackage[portuges,brazil]{babel}
\usepackage[utf8]{inputenc}

\usepackage{indentfirst}
\usepackage{ae}
\usepackage{harvard}
\usepackage{amssymb,fancyhdr,fancybox,epsfig,psfrag,amsmath,tabularx}
\usepackage[paperwidth=8.5in,paperheight=11in,hmargin={25mm,20mm},vmargin={20mm,20mm}]{geometry} %tamanho letter
\usepackage{multirow}
\usepackage[table]{xcolor}

\usepackage[color]{showkeys}
\definecolor{refkey}{rgb}{0.39,0.58,1}
\definecolor{labeled}{rgb}{1,0,0}
\usepackage[Lenny]{fncychap}
\setlength{\headheight}{15pt}

%=========================================== Headers =========================================
\renewcommand{\chaptermark}[1]{\markboth{\chaptername\ \thechapter. \ #1}{ }}
\renewcommand{\sectionmark}[1]{\markright{\thesection. \ #1}}
\fancyhead{}
\fancyfoot{}
\fancyhead[LO]{\nouppercase{\leftmark}}
\fancyhead[RO]{\thepage}

\input{macros}

\begin{document}

\pagenumbering{roman}
\pagestyle{plain}

%================================================================================================
%====================================== FOLHA DE ROSTO ==========================================
%================================================================================================

\begin{center}
\large Universidade Estadual de Campinas\\
Faculdade de Engenharia Elétrica e de Computação
\end{center}

\vspace*{1.5cm}
\begin{center}
\large Francisco Helder Candido dos Santos Filho
\end{center}


\vspace*{2.3cm}

\begin{center}
{\sc  Uma Modelo para Comunicação Machine-To-Machine em Redes IoT Heterogêneas}
\end{center}

\vspace*{3.0cm}

\begin{flushright}
\begin{minipage}{9.0cm}
Qualificação apresentada à Faculdade de Engenharia Elétrica e de Computação como  parte dos requisitos exigidos para a obtenção do título de Doutor em Engenharia Elétrica. Área de concentração: Telecomunicações. 

\vspace*{0.5cm}
Orientador: Paulo Cardieri

\end{minipage}
\end{flushright}

\null \vfill

\vspace*{0.5cm}

\begin{center}
Campinas\\2018
\end{center}

\baselineskip 1.1 \baselineskip

\input{res_abstract}


%=============================== lista de tabelas e figuras ==========================
\listoffigures

\listoftables

%============================ acrônimos, símbolos e notações ========================
\input{acro_notacao.tex}

%=============================== sumário =============================================
\tableofcontents


\pagestyle{fancy}

\chapter{Objetivos e Justificativa do Projeto de Pesquisa}



\chapter{Revis\~{a}o Bibliogr\'{a}fica}

Exemplo de citação: \cite{Lya92}

\chapter{Metodologia Utilizada}

Exemplo de citação: \cite{holler14}
\chapter{Plano de Trabalho e Cronograma}

\section*{Plano de Trabalho}

\section*{Cronograma}


%Tabela
\begin{table}[h]
  \centering
	\caption{table}{Cronograma de Atividades.}
	\begin{tabular}{|l|r||l|l||l|l|l|l||l|l|l|l||l|l|l|l|}
		\hline
		\multirow{2}{*}{\textbf{Atividades}} &\footnotesize\textit{Ano}\normalsize & \multicolumn{2}{c||}{2016} & \multicolumn{4}{c||}{2017} & \multicolumn{4}{c||}{2018} & \multicolumn{4}{c|}{2019} \\
		\cline{2-16}
		&\footnotesize \textit{Trimestre}\footnotesize & \scriptsize 3 &\scriptsize 4 &\scriptsize 1 &\scriptsize 2 &\scriptsize 3 &\scriptsize 4 &\scriptsize 1 &\scriptsize 2 &\scriptsize 3 &\scriptsize 4 &\scriptsize 1 &\scriptsize 2 &\scriptsize 3 &\scriptsize 4 \\
		\hline \hline 
	  \multicolumn{2}{|l||}{\footnotesize Créditos em disciplinas} & \cellcolor{gray} & \cellcolor{gray} &  &  & \cellcolor{gray} & \cellcolor{gray} &  &  &  &  &  &  &  & \\
		\hline
		\multicolumn{2}{|l||}{\footnotesize Levantamento bibliográfico} & \cellcolor{gray} & \cellcolor{gray} & \cellcolor{gray} & \cellcolor{gray}  & \cellcolor{gray} & \cellcolor{gray} &  &  &  &  &  &  &  &  \\
		\hline
	\end{tabular}
	\label{tab:cronograma}
\end{table}


\chapter{Resultados e Conclus\~{o}es Parciais}


\subsection*{Resultados Parciais}


\subsection*{conclus\c{c}\~{o}es Parciais}







%=============================== Bibliografia ===============================================
\addcontentsline{toc}{chapter}{Bibliografia}
\renewcommand{\bibname}{Bibliografia}
\markboth{Bibliografia}{Bibliografia}

\bibliographystyle{dcu}
\bibliography{meubib}

\end{document}
